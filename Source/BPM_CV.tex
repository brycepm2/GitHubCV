%%%%%%%%%%%%%%%%%%%%%%%%%%%%%%%%%%%%%%%%%
% Medium Length Professional CV
% LaTeX Template
% Version 2.0 (8/5/13)
%
% This template has been downloaded from:
% http://www.LaTeXTemplates.com
%
% Original author:
% Trey Hunner (http://www.treyhunner.com/)
%
% Important note:
% This template requires the resume.cls file to be in the same directory as the
% .tex file. The resume.cls file provides the resume style used for structuring the
% document.
%
%%%%%%%%%%%%%%%%%%%%%%%%%%%%%%%%%%%%%%%%%

%----------------------------------------------------------------------------------------
%	PACKAGES AND OTHER DOCUMENT CONFIGURATIONS
%----------------------------------------------------------------------------------------

\documentclass{resume} % Use the custom resume.cls style

\usepackage[left=0.5in,top=0.5in,right=0.5in,bottom=0.5in]{geometry} % Document margins
\newcommand{\tab}[1]{\hspace{.025\textwidth}\rlap{#1}}
\newcommand{\itab}[1]{\hspace{0em}\rlap{#1}}
\name{Bryce Mazurowski} % Your name
% NOTE: I don't want to post my number on GitHub
\address{Chicago, IL 60654} % Your address
\address{brycepm2@gmail.com} % Your phone number and email

\begin{document}

%----------------------------------------------------------------------------------------
%	EDUCATION SECTION
%----------------------------------------------------------------------------------------

\begin{rSection}{Education}

  % UIUC
{\bf The University of Illinois at Urbana-Champaign} \hfill {Overall GPA: 3.79/4.00} 
% PhD
\\ Doctor of Philosophy, Civil Engineering $|$ Structural Engineering \& Mechanics \hfill {\em Aug 2019 - Dec 2024}
\item \vspace{-0.1in} \tab{Dissertation: \textit{A Multiscale Computational Framework for Ceramic Composite}}
\item \vspace{-0.125in} \tab{\tab{ \textit{Structures with Localized Material Nonlinearity (2024)}}}
% Master
\\ Master of Science, Civil Engineering $|$ Structural Engineering  \hfill
{\em Aug 2017 - May 2019}
\item \vspace{-0.1in} \tab{Thesis: \textit{An anisotropic displacement correlation method for extraction}} 
\item \vspace{-0.125in} \tab{\tab{ \textit{of stress intensity factors from three dimensional fractures (2019)}}}\\
% UB Bachelor
{\bf The State University of New York at Buffalo} \hfill { Overall GPA: 3.88/4.00 } 
\\ Bachelor of Science, \textit{summa cum laude}, Civil Engineering \hfill { \em August 2015 - May 2017}
%Minor in Linguistics \smallskip \\
%Member of Eta Kappa Nu \\
%Member of Upsilon Pi Epsilon \\

\end{rSection}
%----------------------------------------------------------------------------------------
%	TECHNICAL STRENGTHS SECTION
%----------------------------------------------------------------------------------------

\begin{rSection}{Technical Strengths}

\begin{tabular}{ @{} >{\bfseries}l @{\hspace{5pt}} r }
Programming Languages &  C++, Tcl, Python, MATLAB, Wolfram Language \\
Software \& Tools & PyTorch, LaTeX, OpenMP, MPI, Abaqus, Patran/NASTRAN \\
Areas of Focus & Damage \& Fracture Mechanics, Multiphysics Simulation, Composite Materials \\
\end{tabular}

\end{rSection}

%----------------------------------------------------------------------------------------
%	WORK EXPERIENCE SECTION
%----------------------------------------------------------------------------------------

\begin{rSection}{Experience}

\begin{rSubsection}{The University of Illinois at Urbana-Champaign}{Aug 2017 - Dec 2024}{Research Assistant}{Advisor: Prof. C. Armando Duarte}
\item Develop and implement algorithms for mathematical models in solid mechanics research code
\item Design and develop material model software package for use with Abaqus
\item Perform tests to verify and validate software tools in a finite element analysis code
\item Maintain functionality of large, evolving C++ research code across multiple operating systems
\item Create tools in scripting languages to optimize repetitive workflows
\item Present research at conferences, suggest future work, propose applications of both
\end{rSubsection}

%------------------------------------------------

\begin{rSubsection}{Boeing Research \& Technology}{Summer 2021  \& 2022}{Graduate Researcher}{}
%
\item Planned a multi-step computational project based on experimental phenomenon
    %
\item Developed and tested material models for high-speed applications in cutting-edge commercial tools
    %
\item Documented working and learning processes of model development to support future endeavors
    %
\item Conducted and communicated strength evaluation of an aerobody preceding flight testing
    %
\item Reviewed aerobody design drawings for accuracy and clarity before sending to manufacturer
    %
\end{rSubsection}
%------------------------------------------------

\begin{rSubsection}{Universal Technology Company}{Summer 2019 \& 2020}{Associate Project Engineer}{Wright Patterson Air Force Base, OH}
\item Perform parametric studies to assess factors that influence fracture behavior
\item Utilize multi-scale structural analysis tools
\item Explore existing models for damage progression in anisotropic materials
\item Implement existing numerical models in research code and verify their accuracy
\item Study effects of material anisotropy on damage progression
\item Develop a research plan for a targeted project utilizing state of the art methods
\end{rSubsection}

\end{rSection}

%----------------------------------------------------------------------------------------

\begin{rSection}{Honors \& Awards}{}
\begin{rSubsection}{}{}{}{}
\item Jacob Karol Fellowship, 2019-2020 academic year at the University of Illinois at Urbana-Champaign
\item Structural Engineering Department Conference Travel Fellowship Award 2019 \& 2020
\item Alfredo \& Myrtle Mae Ang Fellowship, 2017-2018 academic year at the University of Illinois at Urbana-Champaign
\item Dean's List all semesters at the University at Buffalo
\item 2016 Julian Snyder Endowment Fund Student Scholarship Award from the Buffalo Chapter of the American Society of Civil Engineers
\end{rSubsection}
\end{rSection}

%----------------------------------------------------------------------------------------

\begin{rSection}{Extracurricular \& Volunteer Activities}{}
\begin{rSubsection}{}{}{}{}
\item Organized Fall 2019 Qualifying Exam Review Session for the structural engineering exam at the University of Illinois at Urbana-Champaign
\item Member of the Structural Engineering Graduate Student Organization at the University of Illinois at Urbana-Champaign
\end{rSubsection}
\end{rSection}


%----------------------------------------------------------------------------------------

\nobibliography{JournalAbbrev,references_BPM,presentations_BPM}
\bibliographystyle{unsrt}
\begin{rSection}{Publications}{}
\begin{rSubsection}{}{}{}{}
\item \bibentry{2024_Bryce_GFEMgl_CMCBearing}
\item \bibentry{2024_Bryce_GFEMgl_Composites}
\item \bibentry{Mazurowski2022}
\item \bibentry{2020_Bryce_DCM_Aniso}
\item \bibentry{2020_Bryce_HO_SGFEM_SIF_3D}
\item \bibentry{2019_AlfredoSR_SGFEM_pFEM_3D}
\item \bibentry{2019_Bryce_Mazurowski_Ortho_DCM_Thesis}
\end{rSubsection}
\end{rSection}

%----------------------------------------------------------------------------------------

\begin{rSection}{Presentations}{}
\begin{rSubsection}{}{}{}{}
\item \bibentry{EMI2024}
\item \bibentry{USNCCM17}
\item \bibentry{MFEM2022}
\item \bibentry{JANNAF2022}
\item \bibentry{USNCCM16}
\item \bibentry{USNCCM15}
\end{rSubsection}
\end{rSection}


\end{document}
