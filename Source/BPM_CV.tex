%%%%%%%%%%%%%%%%%%%%%%%%%%%%%%%%%%%%%%%%%
% Medium Length Professional CV
% LaTeX Template
% Version 2.0 (8/5/13)
%
% This template has been downloaded from:
% http://www.LaTeXTemplates.com
%
% Original author:
% Trey Hunner (http://www.treyhunner.com/)
%
% Important note:
% This template requires the resume.cls file to be in the same directory as the
% .tex file. The resume.cls file provides the resume style used for structuring the
% document.
%
%%%%%%%%%%%%%%%%%%%%%%%%%%%%%%%%%%%%%%%%%

%----------------------------------------------------------------------------------------
%	PACKAGES AND OTHER DOCUMENT CONFIGURATIONS
%----------------------------------------------------------------------------------------

\documentclass{resume} % Use the custom resume.cls style

\usepackage[left=0.5in,top=0.5in,right=0.5in,bottom=0.5in]{geometry} % Document margins
\newcommand{\tab}[1]{\hspace{.025\textwidth}\rlap{#1}}
\newcommand{\itab}[1]{\hspace{0em}\rlap{#1}}
\name{Bryce Mazurowski} % Your name
% NOTE: I don't want to post my number on GitHub
\address{Chicago, IL 60654} % Your address
\address{brycepm2@gmail.com} % Your phone number and email

\begin{document}

%----------------------------------------------------------------------------------------
%	EDUCATION SECTION
%----------------------------------------------------------------------------------------

\begin{rSection}{Education}

  % UIUC
{\bf The University of Illinois at Urbana-Champaign} \hfill {Overall GPA: 3.79/4.00} 
% PhD
\\ Doctor of Philosophy, Civil Engineering $|$ Structural Engineering \& Mechanics \hfill {\em Aug 2019 - Dec 2024}
% Master
\\ Master of Science, Civil Engineering $|$ Structural Engineering  \hfill
{\em Aug 2017 - May 2019}

% UB Bachelor
{\bf The State University of New York at Buffalo} \hfill { Overall GPA: 3.88/4.00 } 
\\ Bachelor of Science, \textit{summa cum laude}, Civil Engineering \hfill { \em August 2015 - May 2017}

\end{rSection}
%----------------------------------------------------------------------------------------
%	TECHNICAL STRENGTHS SECTION
%----------------------------------------------------------------------------------------

\begin{rSection}{Technical Strengths}

\begin{tabular}{ @{} >{\bfseries}l @{\hspace{5pt}} r }
Programming Languages &  C++, Tcl, Python, MATLAB, Wolfram Language \\
Software \& Tools & PyTorch, LaTeX, OpenMP, MPI, Abaqus, Patran/NASTRAN \\
Areas of Focus & Damage \& Fracture Mechanics, Multiphysics Simulation, Composite Materials \\
\end{tabular}

\end{rSection}

%----------------------------------------------------------------------------------------
%	WORK EXPERIENCE SECTION
%----------------------------------------------------------------------------------------

\begin{rSection}{Experience}

\begin{rSubsection}{The University of Illinois at Urbana-Champaign}{Aug 2017 - Dec 2024}{Research Assistant}{Advisor: Prof. C. Armando Duarte}
\item Research and apply state-of-the-art solid mechanics methods to solve engineering problems 
\item Develop and implement analysis algorithms and tools for finite element research code
\item Design and develop material model software package for use with Abaqus
\item Perform numerical tests to verify and validate software tools in a finite element analysis code
\item Create tools in scripting languages to optimize workflows
\item Present research at conferences, suggest future work, propose applications of both
\end{rSubsection}

%------------------------------------------------

\begin{rSubsection}{Boeing Research \& Technology}{Summer 2021  \& 2022}{Graduate Researcher}{}
%
\item Planned a multi-step computational project based on experimental phenomenon
    %
\item Developed and tested material models for high-speed applications in cutting-edge commercial tools
    %
\item Documented working and learning processes of model development to support future endeavors
    %
\item Conducted and communicated strength evaluation of an aerobody preceding flight testing
    %
\item Reviewed aerobody design drawings for accuracy and clarity before sending to manufacturer
    %
\end{rSubsection}

\end{rSection}

%----------------------------------------------------------------------------------------

\nobibliography{JournalAbbrev,references_BPM,presentations_BPM}
\bibliographystyle{unsrt}
\begin{rSection}{Publications (Title. \textit{Journal.} DOI.)}{}
\begin{rSubsection}{}{}{}{}
\item \bibentry{2024_Bryce_GFEMgl_CMCBearing}
\item \bibentry{2024_Bryce_GFEMgl_Composites}
\item \bibentry{Mazurowski2022}
\item \bibentry{2020_Bryce_DCM_Aniso}
\item \bibentry{2020_Bryce_HO_SGFEM_SIF_3D}
\item \bibentry{2019_AlfredoSR_SGFEM_pFEM_3D}
\item \bibentry{2019_Bryce_Mazurowski_Ortho_DCM_Thesis}
\end{rSubsection}
\end{rSection}

%----------------------------------------------------------------------------------------

\end{document}
